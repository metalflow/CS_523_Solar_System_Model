% !TEX TS-program = pdflatex
% !TEX encoding = UTF-8 Unicode

% This is a simple template for a LaTeX document using the "article" class.
% See "book", "report", "letter" for other types of document.

\documentclass[11pt]{article} % use larger type; default would be 10pt

\usepackage[utf8]{inputenc} % set input encoding (not needed with XeLaTeX)

%%% Examples of Article customizations
% These packages are optional, depending whether you want the features they provide.
% See the LaTeX Companion or other references for full information.

%%% PAGE DIMENSIONS
\usepackage{geometry} % to change the page dimensions
\geometry{a4paper} % or letterpaper (US) or a5paper or....
% \geometry{margin=2in} % for example, change the margins to 2 inches all round
% \geometry{landscape} % set up the page for landscape
%   read geometry.pdf for detailed page layout information

\usepackage{graphicx} % support the \includegraphics command and options

% \usepackage[parfill]{parskip} % Activate to begin paragraphs with an empty line rather than an indent

%%% PACKAGES
\usepackage{authblk} %for more author information
\usepackage{hyperref} % Optional: for clickable links
\usepackage{booktabs} % for much better looking tables
\usepackage{array} % for better arrays (eg matrices) in maths
\usepackage{paralist} % very flexible & customisable lists (eg. enumerate/itemize, etc.)
\usepackage{verbatim} % adds environment for commenting out blocks of text & for better verbatim
\usepackage{subfig} % make it possible to include more than one captioned figure/table in a single float
% These packages are all incorporated in the memoir class to one degree or another...

%%% HEADERS & FOOTERS
\usepackage{fancyhdr} % This should be set AFTER setting up the page geometry
\pagestyle{fancy} % options: empty , plain , fancy
\renewcommand{\headrulewidth}{0pt} % customise the layout...
\lhead{}\chead{}\rhead{}
\lfoot{}\cfoot{\thepage}\rfoot{}

%%% SECTION TITLE APPEARANCE
\usepackage{sectsty}
\allsectionsfont{\sffamily\mdseries\upshape} % (See the fntguide.pdf for font help)
% (This matches ConTeXt defaults)

%%% ToC (table of contents) APPEARANCE
\usepackage[nottoc,notlof,notlot]{tocbibind} % Put the bibliography in the ToC
\usepackage[titles,subfigure]{tocloft} % Alter the style of the Table of Contents
\renewcommand{\cftsecfont}{\rmfamily\mdseries\upshape}
\renewcommand{\cftsecpagefont}{\rmfamily\mdseries\upshape} % No bold!

%%% END Article customizations

%%% The "real" document content comes below...

\title{Solar System Genereation and Modelling}
\author[1]{Cole Risch}
\affil[1]{\href{mailto:crisch@pdx.edu}{crisch@pdx.edu}} % Assigns email to author 1
%\date{} % Activate to display a given date or no date (if empty),
         % otherwise the current date is printed 

\begin{document}
\maketitle

\section{Project Topic Area}
	This project is meant to provide me with some experience generating 3D models in rust and animating their movements.  If possible these animations would be driven by real physics.  Inital rendering framework will be cribbed from the ``3d\_shapes'' example in the Bevy Repository.

\subsection{Known Problems}
	I will NOT be making any attempt to model n-body physics here.  Minimum implementation will be ``on rails'' where planets and planetoids follow a pre-scripted orbit regardless of any physics involved.  Maximum implementaiton would likely involve ``sphere of influence'' implementation of 2 body physics.  This sort of implementation computes the radius at which the gravitational pull of the sun and the gravitational pull of the planet are equal (a Lagrange point).  All small objects in ``deep space'' use the Sun as their only gravitational force, and all small objects crossing one of the Lagrange radii would then use the repective planet as a gravitational force.
\section{Project Vision}
	This project will be implemented using the BEVY framework.  This should absolutely qualify for the usage of ``one or more Rust crates.''  The Bevy framework enforeces a modular code style, so this shouldn't be a problem.
	Realistically, this project *should* evolve through several stages:\\
\begin{enumerate}
    \item  Initial rendering of a Large Sun and one or more smaller planets such that no body intersects any other body (no planets inside the sun, no planets colliding with one another)
    \item animation of the Large Sun and one or more planets rotating about their own axis at different speeds
    \item animation of one or more planets rotating about the Large Sun at different speeds (bonus points for retrograde, elliptical, and orbital incliination)
    \item adding controls to the simulation that allow it to be stopped, started, and reset
    \item adding controls to the simulation that control the number of planets to be created, as well as the radius of the Sun
    \item other controls to be implemented as possible.  For instance, it would be nice to be able to control rotation, radius and orbital speed on a per-planet basis, but I am not certain that is possible.
    \item attempting to implement actual physics for orbits based upon an intial mass of the solar system and then assigning masses to all bodies based on accepted astronomical distributions (e.g. solar mass being 98-99.9\% of the mass of the whole solar system)
    \item attempt to a random texture map to any created planet
    \item attempt to render complex planetoids like Oort cloud, asteroid belts, or moons
\end{enumerate}

	Please note that any item in the previous list that begins with the word ``attempt'' implies a task that I am not certain I will be able to accomplish.

\section{GIT Repo location}
{\href{https://github.com/metalflow/CS\_523\_Solar\_System\_Model.git}{https://github.com/metalflow/CS\_523\_Solar\_System\_Model.git}}

\end{document}
